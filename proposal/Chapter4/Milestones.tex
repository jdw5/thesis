\chapter[Milestones]{Research Plan}

\label{Chap:Milestones}

\section{Milestones}
To track the development of the project, a set of milestones have been established with consideration of assessable dates. 
Table \ref{tab:milestones} below outlines the milestones for the work and what is required for each phase.
Emboldened text indicates an assessable milestone.


\begin{center}
    \small
    \begin{longtable}{p{0.25\linewidth}p{0.5\linewidth}p{0.15\linewidth}}
        \caption{Timeline of milestones.} \label{tab:milestones} \\
        \toprule
        Milestone & Description & Date \\
        \midrule
        \endfirsthead
        
        \caption{Timeline of milestones. (continued)} \\
        \toprule
        Milestone & Description & Date \\
        \midrule
        \endhead
        
        \bottomrule
        \multicolumn{3}{r}{\textit{Continued on next page}} \\
        \endfoot
        
        \bottomrule
        \endlastfoot
        
        \textbf{Project Proposal}	& Complete the project proposal and literature review.                                              & 23/4 \\
        Configure hardware			& Implement NEORV32 processor on FPGA, connect to external components through wishbone interface.   & 23/4 \\
        \textbf{Seminar}            & Present the current project status.                                                               & 23/4 \\
        Image detection algorithm   & Select and implement image detection algorithm in hardware.                                       & 23/4 \\
        VGA driver                  & Develop driver and interface to display processed image on VGA monitor.                           & 23/4 \\
        Benchmark                   & Benchmark system performance against related works.                                               & 23/4 \\
        \textbf{Demo}               & Demonstrate the complete project.                                                                 & 23/4 \\
        \textbf{Thesis}             & Document and write project results.                                                               & 23/4 \\
    \end{longtable}
\end{center}


\nopagebreak


\section{Risk Assessment}
This project is conducted in the low-risk laboratory covered by general OHS laboratory rules, and in a home setting. 
There are no hazardous materials or dangerous equipment used in the project, and the risk of injury is negligible.
The only risk to the project is hardware failure or proprietary software issues, which can be mitigated by using open-source software and hardware, and regular backups of the project files.
Redundancies are also in place for hardware failure, as two FPGA developments boards are designated for use.