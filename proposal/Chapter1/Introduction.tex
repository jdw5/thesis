\chapter[Introduction]{Introduction}
\label{Chap:Motivation}

\section{Motivation}

In recent years, the demand for efficient and versatile image processing systems has surged across various domains, including surveillance, medical imaging, autonomous vehicles, and more. 
Field Programmable Gate Arrays (FPGAs) have emerged as a prominent solution due to their reconfigurability, parallel processing capabilities, and low power consumption.
Hence, image processing is a good candidate for the application of parallel processing due to the algorithm complexity and large volume of data. \cite{Efficient}

One significant aspect of image processing systems is the choice of the underlying processor architecture. 
Traditional approaches often rely on general-purpose CPUs or GPUs to execute image processing tasks. 
However, these architectures may not always offer the best balance of performance, power efficiency, and flexibility for image processing applications. 
In recent years, the RISC-V instruction set architecture (ISA) has gained traction as an open, customizable, and energy-efficient alternative to proprietary processor designs.
These soft processors can be implemented on FPGAs, providing a flexible and scalable platform for image processing tasks.

This thesis focuses on the exploration and implementation of image detection algorithms using a RISC-V processor architecture deployed on an FPGA platform. 
By leveraging the configurability of FPGAs and the energy efficiency of the RISC-V ISA, this research aims to develop a scalable and adaptable solution for real-time image detection tasks.
The data processing capacity is large that the processing speed must be strict to meet the demands of real-time time image transmission \cite{Video}.