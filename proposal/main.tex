\documentclass[instructions]{uqthesis}
%\documentclass[final]{uqthesis} 


%*************************************
% FOR YOUR FINAL THESIS
%*************************************

%IMPORTANT! 
%The default document class (above - line 1 & 2) for the template is \documentclass[instructions]{uqthesis} - this document class will show instructional material and examples relevant to the preliminary material in the compiled PDF preview. THESE INSTRUCTIONS ARE FOR YOUR REFERENCE ONLY AND ARE NOT TO BE INCLUDED IN YOUR FINAL THESIS! 

%To turn off these instructions in your final thesis you MUST use the document class \documentclass[final]{uqthesis} 
%To activate the final thesis document class you must UN-COMMENT THIS DOCUMENT CLASS (remove the % from the start of line 2) and comment out the instructional document class on line 1 (add % to the start of line 1). 

%*************************************
% Introduction to template
%*************************************
%This is The University of Queensland Graduate School Official LaTeX Thesis template.

%Be sure to observe the content of comments within the source code, these are prefaced with a percentage symbol.
%Most important instructions have been CAPITALISED.
%To uncomment an inactive command (if required) remove the % from in front of the command.

%Please see the README for more information.

%This file loads the necessary packages, sets the page styles, and defines required macros.
%Edit this if you are comfortable with LaTeX.

%Other tweaks can be made in uqthesis.cls, but change these at your own risk!

%See README for version.

%You must have the memoir class installed.

% ***************************************************
% LaTeX Packages
% ***************************************************
% This file defines the document design.
% Usually it is not necessary to edit this file, but you can use it to change aspects of the design if you want.

%There are essential packages that are contained within the uqthesis.cls which are integral to the template - These must not be deleted.  A list of these packages can be found in the README.tet file

%The packages below are optional, please add or alter as required.

\usepackage{cite}				 %Allows abbreviated numerical citations.
\usepackage{pdfpages}			 %Allows you to include full-page pdfs.
\usepackage{wrapfig}			 %Lets you wrap text around figures.
\usepackage{bm} 				 %Bolded maths characters.
\usepackage{upgreek}			 %Upright Greek characters.
\usepackage{dsfont}				 %Double-struck fonts.
\usepackage{simplewick}			 %For typesetting Wick contractions.
\usepackage{mathtools}		     %Can be used to fine-tune the maths presentation.	
\usepackage{framed}			     %For boxed text.
\usepackage{microtype}			 %pdfLaTeX will fix your kerning.
\usepackage{marvosym}			 %Include symbols (like the Euro symbol, etc.).
\usepackage{color}				 %Nice for scalable pdf graphics using InkScape.
\usepackage{transparent}	     %Nice for scalable pdf graphics using InkScape.
\usepackage{placeins}			 %Lets you put in a \FloatBarrier to stop figures floating past this command.
\usepackage{mdframed,mdwlist}    %Use these for nice lists (less white space).
\usepackage{graphicx}            %Enhanced support for graphics.
\usepackage{float}               %Improved interface for floating objects. 
\usepackage{longtable}           %Allow tables to flow over page boundaries.
\usepackage{mathdots}            %Changed the basic LaTeX and plain TeX commands.
\usepackage{eucal}               %Font shape definitions to use the Euler script symbols in math mode.
\usepackage{array}               %Extending the array and tabular environments.
\usepackage{stmaryrd}            %The StMary’s Road symbol font.
\usepackage{amsthm}              %St Mary Road symbols for theoretical computer science. 
\usepackage{pifont}              %Access to PostScript standard Symbol and Dingbats fonts.
\usepackage{lipsum}              %Easy access to the Lorem Ipsum dummy text.
\usepackage{enumerate}           %Enumerate with redefinable labels. 
\usepackage[all]{xy}             %This is a special package for drawing diagrams.
\usepackage{amsmath}             %ATypesetting theorems (AMS style).
\usepackage{amssymb}             %Provided an extended symbol collection.
\usepackage[utf8]{inputenc}      %Allowed all displayable utf8 characters to be available as input.
\usepackage{fancyhdr}            %Extensive control of page headers and footers.
\usepackage{blindtext}           %Produced 'blind' text for testing.
\usepackage{tikz}                %To create graphic elements.
\usepackage[figuresright]{rotating}	%Allows large tables to be rotated to landscape.
\usepackage{makecell}
\usepackage{tabularx}
\usepackage{titlesec}


\usetikzlibrary{shapes.geometric, arrows}
%You can add more packages here if you need


%This defines some macros that implement Latin abbreviations
%COMMENT OUT OR DELETE IF UNDESIRED.
\newcommand{\via}{\textit{via}} %Italicised via.
\newcommand{\ie}{\textit{i.e.}} %Literally.
\newcommand{\eg}{\textit{e.g.}} %For example.
\newcommand{\etc}{\textit{etc.}} %So on...
\newcommand{\vv}{\textit{vice versa}} %And the other way around.
\newcommand{\viz}{\textit{viz}.} %Resulting in.
\newcommand{\cf}{\textit{cf}.} %See, or 'consistent with'.
\newcommand{\apr}{\textit{a priori}} %Before the fact.
\newcommand{\apo}{\textit{a posteriori}} %After the fact.
\newcommand{\vivo}{\textit{in vivo}} %In the flesh.
\newcommand{\situ}{\textit{in situ}} %On location.
\newcommand{\silico}{\textit{in silico}} %Simulation.
\newcommand{\vitro}{\textit{in vitro}} %In glass.
\newcommand{\vs}{\textit{versus}} %James \vs{} Pete.
\newcommand{\ala}{\textit{\`{a} la}} %In the manner of...
\newcommand{\apriori}{\textit{a priori}} %Before hand.
\newcommand{\etal}{\textit{et al.}} %And others, with correct punctuation.
\newcommand{\naive}{na\"\i{}ve} %Queen Amidala is young and \naive{}.

% ***************************************************
% Title page
% ***************************************************
%***THESIS TITLE***
%Use Sentence Case (capitalise only the first word and proper nouns).
\title{Image Detection with RISC-V Processor on FPGA}

%***YOUR NAME***
%Do not include initials or middle names. Do not include your supervisor(s)' name(s).
\author{Joshua Wallace}
%***YOUR CURRENT DEGREES***
%Use abbreviations. Do not include the date or location of your degree. Do not include the degree for which this thesis is being submitted.
\currentdegrees{Project Proposal}

%***ORCID ID***
%Add and hyperlink your ORCID

%***YEAR OF SUBMISSION***
\date{\{2024\}}
%***TYPE OF DEGREE***
\submittedfor{Thesis - METR4911}


%***YOUR SCHOOL***
%Use Title Case (capitalise every word which is not a conjunction or preposition).
%See - http://blog.apastyle.org/apastyle/2012/03/title-case-and-sentence-case-capitalization-in-apa-style.html - for help.
\school{School of Information Technology and Electrical Engineering}

\begin{document}

\frontmatter
% Assemble title page
\maketitle
\clearpage

% ***************************************************
% Preface
%****************************************************
\section{Abstract}
\normalfont
%Open abstract.tex to edit
\input{./Abstract/Abstract.tex}

\clearpage
% ***************************************************
\section*{Declaration by author}
%DO NOT EDIT.
\input{./Authordeclaration.tex}

\clearpage
%YOU MUST EDIT THIS DOCUMENT.
% ***************************************************
% PRELIMINARY PAGES
% ***************************************************
% The instructions contained within this part of the thesis template need to be suppressed from the final thesis. There are instructions on how to do this in the MainThesis.tex file.

% To ensure your work is not suppressed with the instructions please add your text only where instructed.


\clearpage
\pagestyle{headings}

\chapter[List of Abbreviations ]{List of Abbreviations}

%If the auto-sizing of the tables annoys you, consider the tabularx package.

%List of abbreviations.
\begin{center}
	\small
	\begin{longtable}{ll}
	\toprule
	Abbreviations & {} \\
	\bottomrule
	FPGA			& Field Programmable Gate Array \\
	ASIC			& Application Specific Integrated Circuit \\
	CPU				& Central Processing Unit \\
	GPU				& Graphical Processing Unit \\
	SoC				& System on Chip \\
	ISA				& Instruction Set Architecture \\
    RISC            & Reduced Instruction Set Computer \\
	CISC            & Complex Instruction Set Computer \\
	IO				& Input Output \\
	RTOS			& Real Time Operating System \\
	CNN			    & Convolutional Neural Network \\
    YOLO            & You Only Look Once \\
	VGA				& Video Graphics Array \\
	\hline 
	\end{longtable}
\end{center}

\clearpage

%***Table of Contents***
%These generate the table of contents, list of figures, and list of tables from items tagged with a \label{} command.
\tableofcontents
	\clearpage
\listoffigures
\listoftables

% \input{./PreliminaryAndBackPages/Symbols.tex} %List of symbols. REMOVE IF NOT NEEDED.

%***End of front matter***

% ***************************************************
% Thesis Content
%****************************************************
\mainmatter

%Each chapter is a separate .tex file. Use \input to load them here, as demonstrated below for Chapter 1 and Chapter 2.
%We recommend keeping each in a separate subfolder, with its accompanying figures, etc. This is how the template is currently structured.
%If you wish to divide your thesis into parts (each containing multiple chapters), us the \part{} command.

%CHAPTER 1
\chapter[Introduction]{Introduction}
\label{Chap:Intro}

\section{Introduction}

Introduction section

%CHAPTER 2
\input{./Chapter2/Chapter2.tex}

% HOW TO ADD ADDITIONAL CHAPTERS
% Step One: Add a new folder called "ChapterX" (X being the chapter number).
% Step Two: Within the folder add a new .tex file by clicking the "New File" button in the Overleaf Menu. Rename the file to a title of your choice.
% Step Three: Copy the Chapter 2 headline and "\input" command located above and insert it below Chapter 2.
% Step Four: Rename the headline to your specific chapter number, change the input command to include the name of the folder you created and the name of the file you created.
% Repeat this process for every chapter.

%CONCLUSION CHAPTER
\input{./Conclusion/Conclusion.tex}

% ***************************************************
% Bibliography
%****************************************************
%CHOOSE YOUR BIB STYLE AND FILE.
%We have included the following two referencing styles for you to use in your thesis. You can add an alternate style if you prefer.

%Style: apalike = this is an (Author, Year) referencing style similar to APA
%Style: elsarticle-num = this is a numbered referencing style that will display the bibliography in citation order

%To use one of the styles provided ensure the % is removed from the start of the line, and the other option is commented out with a % at the start of the line. The style elsarticle-num is active by default.

%\bibliographystyle{apalike}
\bibliographystyle{elsarticle-num}

\bibliography{./References/Bibliography}

%When you have finished your thesis we recommend that you manually fix any errors in your bibliography. 
%To do this, compile, copy the .bbl into a new .tex file and include this here after commenting out the other bibliography commands. Make corrections in that .tex file.

% ***************************************************
% Appendices
%**************************************************** 
%UNCOMMENT THIS SECTION IF YOU ARE USING APPENDICES.
%Simply adapt the same formatting used for other chapters.
\appendix
% If you need appendix in your thesis then consider the following appendix file (you can add more if you need more) otherwise you should not consider it in your main thesis.
\include{./Appendix/Appendix}

% ***************************************************
% Examples
%**************************************************** 
% The following files are only for examples and you should not include them in your final thesis.
\include{./Examples/ExampleOfCitations}
\include{./Examples/ExampleOfCode}
\include{./Examples/ExampleOfEquations}
\include{./Examples/ExampleOfFigures}
\include{./Examples/ExampleOfFlowCharts}
\include{./Examples/ExampleOfTables}


% ***************************************************
% Back Matter
%**************************************************** 
%COMMENT OUT IF YOU DO NOT WISH TO INCLUDE BACK MATTER.
\input{./PreliminaryAndBackPages/Back.tex}

\end{document}
